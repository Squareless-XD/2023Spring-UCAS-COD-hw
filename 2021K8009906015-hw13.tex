% -----------------------------*- LaTeX -*------------------------------
\documentclass[UTF8]{report}
% ------------------------------------------------------------------------
% Packages
% ------------------------------------------------------------------------
\usepackage{adjustbox}
\usepackage{algorithm,algorithmicx}
\usepackage[noend]{algpseudocode}
\usepackage{amsmath,amsfonts,amssymb,bm,amsthm} % 数学宏包、数学字体、数学符号、支持 \mathscr{} 字体、支持粗斜体 \bm{}、数学定理
\usepackage{bigstrut,multirow,rotating} % Excel表格自动导入latex
\usepackage{booktabs}
\usepackage{breqn}
\usepackage{caption}
\usepackage{color} % 支持颜色改变
\usepackage{ctex}
\usepackage{enumitem} % 自定义列表环境
\usepackage{esint} % 支持多种积分算子
\usepackage{extarrows} % 任意长度的箭头
\usepackage{fancyhdr}
\usepackage{fontsize}
\usepackage{fontspec}
\usepackage[body={7in, 9in},left=1in,right=1in]{geometry}
\usepackage{graphicx} % 支持 \includegraphics{} 插图
\usepackage{mathrsfs}
\usepackage{mathtools} % 数学宏包的重要补充
\usepackage[framemethod=TikZ]{mdframed}
\usepackage{nicefrac}
\usepackage{scribe}
\usepackage{subfigure} % 插入子图
\usepackage{tikz,xcolor} % 画图、画 Feynman 图
\usepackage{upgreek} % 数学环境的直立希腊字母
% ------------------------------------------------------------------------
% Macros
% ------------------------------------------------------------------------
%~~~~~~~~~~~~~~~
% Utility latin
%~~~~~~~~~~~~~~~
\newcommand{\ie}{\textit{i.e.}}
\newcommand{\eg}{\textit{e.g.}}
%~~~~~~~~~~~~~~~
% Environment shortcuts
%~~~~~~~~~~~~~~~
\newcommand{\balign}[1]{\ealign{\begin{align}#1\end{align}}}
\newcommand{\baligns}[1]{\ealigns{\begin{align*}#1\end{align*}}}
\newcommand{\bitemize}[1]{\eitemize{\begin{itemize}#1\end{itemize}}}
\newcommand{\benumerate}[1]{\eenumerate{\begin{enumerate}#1\end{enumerate}}}
%~~~~~~~~~~~~~~~
% Text with quads around it
%~~~~~~~~~~~~~~~
\newcommand{\qtext}[1]{\quad\text{#1}\quad}
%~~~~~~~~~~~~~~~
% Shorthand for math formatting
%~~~~~~~~~~~~~~~
\newcommand{\mbb}[1]{\mathbb{#1}}
\newcommand{\mbi}[1]{\boldsymbol{#1}} % Bold and italic (math bold italic)
\newcommand{\mbf}[1]{\mathbf{#1}}
\newcommand{\mc}[1]{\mathcal{#1}}
\newcommand{\mrm}[1]{\mathrm{#1}}
\newcommand{\tbf}[1]{\textbf{#1}}
\newcommand{\tsc}[1]{\textsc{#1}}
% \def\<{{\langle}}
% \def\>{{\rangle}}
\newcommand{\sT}{\sf T}
\newcommand{\grad}{\nabla}
\newcommand{\Proj}{\Pi}
%~~~~~~~~~~~~~~~
% Common sets 定义数集符号
%~~~~~~~~~~~~~~~
\newcommand{\R}{\mathbb{R}}
\newcommand{\Z}{\mathbb{Z}}
\newcommand{\Q}{\mathbb{Q}}
\newcommand{\N}{\mathbb{N}}
\newcommand{\C}{\mathbb{C}}
\newcommand{\reals}{\mathbb{R}} % Real number symbol
\newcommand{\integers}{\mathbb{Z}} % Integer symbol
\newcommand{\rationals}{\mathbb{Q}} % Rational numbers
\newcommand{\naturals}{\mathbb{N}} % Natural numbers
\newcommand{\complex}{\mathbb{C}} % Complex numbers
%~~~~~~~~~~~~~~~
% Common functions
%~~~~~~~~~~~~~~~
\renewcommand{\exp}[1]{\operatorname{exp}\left(#1\right)} % Exponential
\newcommand{\indic}[1]{\mbb{I}\left(#1\right)} % Indicator function
\newcommand{\indicsub}[2]{\mbb{I}_{#2}\left(#1\right)} % Indicator function
\newcommand{\argmax}{\mathop\mathrm{arg\, max}} % Defining math symbols
\newcommand{\argmin}{\mathop\mathrm{arg\, min}}
\renewcommand{\arccos}{\mathop\mathrm{arccos}}
\newcommand{\dom}{\mathop\mathrm{dom}} % Domain
\newcommand{\range}{\mathop\mathrm{range}} % Range
\newcommand{\diag}{\mathop\mathrm{diag}}
\newcommand{\tr}{\mathop\mathrm{tr}}
\newcommand{\abs}{\mathop\mathrm{abs}}
\newcommand{\card}{\mathop\mathrm{card}}
\newcommand{\sign}{\mathop\mathrm{sign}}
\newcommand{\prox}{\mathrm{prox}} % prox
\newcommand{\rank}[1]{\mathrm{rank}(#1)}
\newcommand{\supp}[1]{\mathrm{supp}(#1)}
\newcommand{\norm}[1]{\lVert#1\rVert}
%~~~~~~~~~~~~~~~
% Common probability symbols
%~~~~~~~~~~~~~~~
\newcommand{\family}{\mathcal{P}} % probability family / statistical model
\newcommand{\iid}{\stackrel{\mathrm{iid}}{\sim}}
\newcommand{\ind}{\stackrel{\mathrm{ind}}{\sim}}
\newcommand{\E}{\mathbb{E}} % Expectation symbol
\newcommand{\Earg}[1]{\E\left[#1\right]}
\newcommand{\Esubarg}[2]{\E_{#1}\left[#2\right]}
\renewcommand{\P}{\mathbb{P}} % Probability symbol
\newcommand{\Parg}[1]{\P\left(#1\right)}
\newcommand{\Psubarg}[2]{\P_{#1}\left[#2\right]}
% \newcommand{\Cov}{\mrm{Cov}} % Covariance symbol
% \newcommand{\Covarg}[1]{\Cov\left[#1\right]}
% \newcommand{\Covsubarg}[2]{\Cov_{#1}\left[#2\right]}
% \newcommand{\model}{\mathcal{P}} % probability family / statistical model
%~~~~~~~~~~~~~~~
% Distributions
%~~~~~~~~~~~~~~~
% \newcommand{\Gsn}{\mathcal{N}}
% \newcommand{\Ber}{\textnormal{Ber}}
% \newcommand{\Bin}{\textnormal{Bin}}
% \newcommand{\Unif}{\textnormal{Unif}}
% \newcommand{\Mult}{\textnormal{Mult}}
% \newcommand{\NegMult}{\textnormal{NegMult}}
% \newcommand{\Dir}{\textnormal{Dir}}
% \newcommand{\Bet}{\textnormal{Beta}}
% \newcommand{\Gam}{\textnormal{Gamma}}
% \newcommand{\Poi}{\textnormal{Poi}}
% \newcommand{\HypGeo}{\textnormal{HypGeo}}
% \newcommand{\GEM}{\textnormal{GEM}}
% \newcommand{\BP}{\textnormal{BP}}
% \newcommand{\DP}{\textnormal{DP}}
% \newcommand{\BeP}{\textnormal{BeP}}
% \newcommand{\Exp}{\textnormal{Exp}}
%~~~~~~~~~~~~~~~
% Theorem-like environments
%~~~~~~~~~~~~~~~
% \theoremstyle{definition}
% \newtheorem{definition}{Definition}
% \newtheorem{example}{Example}
% \newtheorem{problem}{Problem}
% \newtheorem{lemma}{Lemma}
%~~~~~~~~~~~~~~~
% 组合数学的模板和作业里用到的一些宏包和自定义命令
%~~~~~~~~~~~~~~~
\renewcommand{\emph}[1]{\begin{kaishu}#1\end{kaishu}}
\newcommand{\falfac}[1]{^{\underline{#1}}}
\newcommand{\binomfrac}[2]{\frac{#1^{\underline{#2}}}{#2!}}
\newcommand{\ceil}[1]{\left\lceil #1 \right\rceil}
\newcommand{\floor}[1]{\left\lfloor #1 \right\rfloor}
\newcommand{\suminfty}[2]{\sum_{#1=#2}^{\infty}}
\newcommand{\suminftyk}[0]{\sum_{k=0}^{\infty}}
\newcommand{\sumint}[3]{\sum_{#1=#2}^{#3}}
\newcommand{\sumintk}[2]{\sum_{k=#1}^{#2}}
\newcommand{\suminti}[2]{\sum_{i=#1}^{#2}}
%~~~~~~~~~~~~~~~
% 定义新命令
%~~~~~~~~~~~~~~~
\newcommand*{\unit}[1]{\mathop{}\!\mathrm{#1}}
\newcommand*{\dif}{\mathop{}\!\mathrm{d}}%微分算子 d
\newcommand*{\pdif}{\mathop{}\!\partial}%偏微分算子
\newcommand*{\cdif}{\mathop{}\!\nabla}%协变导数、nabla 算子
\newcommand*{\laplace}{\mathop{}\!\Delta}%laplace 算子
\newcommand*{\deriv}[2]{\frac{\mathrm{d} #1}{\mathrm{d} {#2}}}
\newcommand*{\derivh}[3]{\frac{\mathrm{d}^{#1} #2}{\mathrm{d} {#3^{#1}}}}
\newcommand*{\pderiv}[2]{\frac{\partial #1}{\partial {#2}}}
\newcommand*{\pderivh}[3]{\frac{\partial^{#1} #2}{\partial {#3^{#1}}}}
\newcommand*{\dderiv}[2]{\dfrac{\mathrm{d} #1}{\mathrm{d} {#2}}}
\newcommand*{\dderivh}[3]{\dfrac{\mathrm{d}^{#1} #2}{\mathrm{d} {#3^{#1}}}}
\newcommand*{\dpderiv}[2]{\dfrac{\partial #1}{\partial {#2}}}
\newcommand*{\dpderivh}[3]{\dfrac{\partial^{#1} #2}{\partial {#3^{#1}}}}
\newcommand{\me}[1]{\mathrm{e}^{#1}}%e 指数
\newcommand{\mi}{\mathrm{i}}%虚数单位
% \newcommand{\mc}{\mathrm{c}}%光速 定义与mathcal冲突
\newcommand{\red}[1]{\textcolor{red}{#1}}
\newcommand{\blue}[1]{\textcolor{blue}{#1}}
% \newcommand{\Rome}[1]{\setcounter{rome}{#1}\Roman{rome}}
%~~~~~~~~~~~~~~~
% 公式环境中箭头符号的简写
%~~~~~~~~~~~~~~~
\newcommand{\ra}{\rightarrow}
\newcommand{\Ra}{\Rightarrow}
\newcommand{\la}{\leftarrow}
\newcommand{\La}{\Leftarrow}
\newcommand{\lra}{\leftrightarrow}
\newcommand{\Lra}{\Leftrightarrow}
\newcommand{\lgla}{\longleftarrow}
\newcommand{\Lgla}{\Longleftarrow}
\newcommand{\lgra}{\longrightarrow}
\newcommand{\Lgra}{\Longrightarrow}
\newcommand{\lglra}{\longleftrightarrow}
\newcommand{\Lglra}{\Longleftrightarrow}
%~~~~~~~~~~~~~~~
% 一些数学的环境设置
%~~~~~~~~~~~~~~~
% \newcounter{counter_exm}\setcounter{counter_exm}{1}
% \newcounter{counter_prb}\setcounter{counter_prb}{1}
% \newcounter{counter_thm}\setcounter{counter_thm}{1}
% \newcounter{counter_lma}\setcounter{counter_lma}{1}
% \newcounter{counter_dft}\setcounter{counter_dft}{1}
% \newcounter{counter_clm}\setcounter{counter_clm}{1}
% \newcounter{counter_cly}\setcounter{counter_cly}{1}
% \newtheorem{theorem}{{\hskip 1.7em \bf 定理}}
% \newtheorem{lemma}[theorem]{\hskip 1.7em 引理}
% \newtheorem{proposition}[theorem]{Proposition}
% \newtheorem{claim}[theorem]{\hskip 1.7em 命题}
% \newtheorem{corollary}[theorem]{\hskip 1.7em 推论}
% \newtheorem{definition}[theorem]{\hskip 1.7em 定义}
\newcommand{\problem}[1]{{\setlength{\parskip}{10pt}\noindent \bf{#1}}}
\newenvironment{solution}{{\noindent\hskip 2em \bf 解 \quad}}{}
\renewenvironment{proof}{{\setlength{\parskip}{7pt}\noindent\hskip 2em \bf 证明 \quad}}{\hfill$\qed$\par}
% \newenvironment{example}{{\noindent\hskip 2em \bf 例 \arabic{counter_exm}\quad}}{\addtocounter{counter_exm}{1}\par}
% \newenvironment{concept}[1]{{\bf #1\quad} \begin{kaishu}} {\end{kaishu}\par}
%~~~~~~~~~~~~~~~
% 本.tex文档中特殊定义命令
%~~~~~~~~~~~~~~~
\newcommand{\cdclass}[2]{[#1]_{\text{#2}}}

% ----------------------------------------------------------------------
% Header information
% ------------------------------------------------------------------------

\begin{document}

\course{B0911006Y-01} 			%optional
\coursetitle{Computer Organization and Design}	%optional
\semester{2023 Spring}		%optional
\lecturer{Ke Zhang}	%optional
\scribe{吉骏雄}		%required
\lecturenumber{13}			%required (must be a number)
\lecturedate{June 21}	%required (omit year)

\maketitle

% ----------------------------------------------------------------------
% Body of the document
% ------------------------------------------------------------------------


\textbf{《计算机组成原理》 (唐朔飞版) 课后习题4.28, 4.29, 4.32, 4.38, 4.39}

\problem{4.28} 设主存容量为256K字, Cache容量为2K字, 块长为4.
\begin{enumerate}[label=(\arabic*)]
    \item 设计Cache地址格式,  Cache 中可装入多少块数据? 
    \item 在直接映射方式下, 设计主存地址格式. 
    \item 在四路组相联映射方式下, 设计主存地址格式. 
    \item 在全相联映射方式下, 设计主存地址格式. 
    \item 若存储字长为$32$位, 存储器按字节寻址, 写出上述三种映射方式下主存的地址格式. 
\end{enumerate}

\begin{solution}
    \begin{enumerate}[label=(\arabic*)]
        \item Cache一共有$2\mrm{K} = 2^{11}$个字, 块长为$4 = 2^2$, 需要$2$位地址编号, 剩余的$11-2=9$位用于标记, 共有$2^9 = 512$个块. 地址格式为前$9$位标记, 后$2$位块内地址.
        \item 主存地址的字长为$\log{256\mrm{K}} = \log{2^{18}} = 18$, 因此除了$9$位CaChe块地址和$2$位块内地址外, 还有$18-11=7$位主存内字块标记. 直接映射方式下, 主存地址格式为前$7$位主存内字块标记, 中间$9$位CaChe块地址, 后$2$位块内地址.
        \item 四路组相联映射方式下, 因为有4个块被放至同一集合的不同路中, 组地址比直接映射少了两个地址字长, 因此主存地址格式为前$9$位主存内字块标记, 中间$7$位组地址, 后$2$位块内地址.
        \item 全相联映射方式下, 因为每个块都可以放在任意一路中, 因此主存地址格式为前$16$位主存内字块标记, 后$2$位块内地址.
        \item 存储字长$32 = 8 \times 4$, 因此一个存储字长需要另外的$\log{4} = 2$个字节来编址. 于是三种结果分别如下:
        
        直接映射方式下, 主存地址格式为前$7$位主存内字块标记, 中间$9$位CaChe块地址, 后$4$位块内地址.

        四路组相联映射方式下, 主存地址格式为前$9$位主存内字块标记, 中间$7$位组地址, 后$4$位块内地址.

        全相联映射方式下, 主存地址格式为前$16$位主存内字块标记, 后$4$位块内地址.
    \end{enumerate}
\end{solution}
    

\problem{4.29} 假设CPU执行某段程序时共访问Cache命中4800次, 访问主存200次, 已知Cache的存取周期是30ns, 主存的存取周期是150ns, 求Cache的命中率以及Cache-主存系统的平均访问时间和效率, 试问该系统的性能提高了多少?

\begin{solution}

    Cache的命中率为$\frac{4800}{4800+200} \times 100\% = 96\%$.
    
    Cache-主存系统的平均访问时间为$96\% \times 30 + 4\% \times 150 = 34.8\unit{ns}$.
    
    Cache-主存系统的效率为$\frac{30}{34.8} \times 100\% = 86.2\%$.
    
    如果没有Cache, 平均访问时间为$150\unit{ns}$.

    因此该系统的性能提高了$\frac{150}{34.8} - 1 = 3.31$倍.
\end{solution}


\problem{4.32} 设某机主存容量为4MB, Cache容量为16KB, 每字块有8个字, 每字32位, 设计一个四路组相联映射 (即Cache每组内共有4个字块) 的Cache组织.
\begin{enumerate}[label=(\arabic*)]
    \item 画出主存地址字段中各段的位数.
    \item 设Cache的初态为空, CPU依次从主存第0, 1, 2, …, 89号单元读出90个字 (主存一次读出一个字) , 并重复按此次序读8次, 间命中率是多少?
    \item 若Cache的速度是主存的6倍, 试问有Cache和无Cache相比, 速度约提高多少倍?
\end{enumerate}

\begin{solution}
    \begin{enumerate}[label=(\arabic*)]
        \item 由于Cache的容量为16KB, 每字块有8个字, 每字32位, 另有$4$路组相联结构, 因此Cache共有$\log 16\mrm{K} - \log 8 - \log \frac{32}{8} - \log{4} = 14-3-2-2 = 7$个字块, 需要$7$位组地址编号, $5$位块内地址编号, 剩余的$22-7-5=10$位用于标记. 主存地址格式为前$10$位主存块标志, 中间$7$位组地址, 后$5$位块内地址.

        \begin{table}[htbp]
            \centering
            \caption{主存地址字段中各段的位数}
            \begin{tabular}{|c|c|c|}
                \hline
                主存块标志 & 组地址 & 块内地址 \bigstrut[t]\\
                10位 & 7位 & 5位 \bigstrut[b]\\
                \hline
            \end{tabular}%
            % \label{tab:addlabel}%
        \end{table}%

        \item Cache的块地址一共有$7$位, 因此一共有$2^7 = 128$个块, 这些块足够把连续的$90$个字存入. 对于第$0, 8, 16, 24, 32, 40, 48, 56, 64, 72, 80, 88$个字的读取时, Cache需要每次从主存中获取$8$个单元(字)的数据, 而其他时候都能保证Cache命中 (包括后续的重复读取, 因为并没有出现需要驱逐Cache数据的情况). 因此Cache的命中率为$\frac{8 \times 90-12}{8 \times 90} \times 100\% = \frac{59}{60} = 98.33\%$.
        
        \item 若Cache的速度是主存的6倍, 试问有Cache和无Cache相比, 速度约提高多少倍?
        
        有Cache的平均读取时间: $\frac{59}{60} \times t + \frac{1}{60} \times 6t = \frac{13}{12} t = 1.083 t \unit{ns}$.

        无Cache的平均读取时间: $6t \unit{ns}$.

        速度提高的倍数: $\frac{6}{\frac{13}{12}} - 1 = 4.538$倍.
    \end{enumerate}
\end{solution}


\problem{4.38} 磁盘组有6片磁盘, 最外两侧盘面可以记录, 存储区域内径22cm, 外径33cm, 道密度为40道/cm, 内层密度为400位/cm, 转速3600r/min.
\begin{enumerate}[label=(\arabic*)]
    \item 共有多少存储面可用?
    \item 共有多少柱面?
    \item 盘组总存储容量是多少?
    \item 数据传输率是多少?
\end{enumerate}

\begin{solution}
    \begin{enumerate}[label=(\arabic*)]
        \item 因为题目说最外两侧盘面可以记录, 因此共有$6 \times 2 = 12$个盘面可用.
        \item 盘面的道密度为$40\unit{\text{道}/cm}$, 由于内径/外径两个直径之间的差值包含同一条磁道两次, 所以有效长度为$\frac{33-22}{2} = 5.5\unit{cm}$, 因此每个盘面共有$5.5 \times 40 = 220$道, 因此也有$220$个柱面.
        \item 每个盘面的内层密度为$400\unit{\text{位}/cm}$, 内层磁道周长为$\pi \times 22 = 69.12 \unit{cm}$, 因此每个磁道的容量为$69.12 \times 400 = 27648\unit{位}$, 因此盘组总存储容量为$27648 \times 12 \times 220 = 72990720 \unit{位} = 9123840\mrm{B}$.
        \item 数据传输率为$\frac{3600}{60} \times 27648 = 1658880 \unit{\text{位}/s} = 207360 \unit{B/s} = 202.5 \unit{KB/s}$.
    \end{enumerate}
\end{solution}


\problem{4.39} 某磁盘存储器转速为3000r/min, 共有4个记录盘面, 每毫米5道, 每道记录信息12288字节, 最小磁道直径为230mm, 共有275道, 求:
\begin{enumerate}[label=(\arabic*)]
    \item 磁盘存储器的存储容量.
    \item 最高位密度 (最小磁道的位密度) 和最低位密度.
    \item 磁盘数据传输率.
    \item 平均等待时间.
\end{enumerate}

\begin{solution}
    \begin{enumerate}[label=(\arabic*)]
        \item 磁盘存储器的存储容量为$4 \times 275 \times 12288 = 13516800 \unit{B} = 13200 \unit{KB} \approx 12.89 \unit{MB}$.
        \item 有效长度为$\frac{275}{5} = 55 \unit{mm}$, 因此外径为$230 + 55 \times 2 = 340 \unit{mm}$.
        
        最高位密度为$\frac{12288 \times 8}{\pi \times 340} \approx 92.033 \unit{位}/\unit{mm}$.
        
        最低位密度为$\frac{12288 \times 8}{\pi \times 230} \approx 136.048 \unit{位}/\unit{mm}$.
        \item 数据传输率为$\frac{3000}{60}  \times 12288 = 614400 \unit{B/\text{min}} = 600 \unit{KB/\text{min}}$.
        \item 平均等待时间为$\frac{1}{2} \times \frac{1}{3000} \times 60 = 0.01 \unit{s} = 10 \unit{ms}$.
    \end{enumerate}
\end{solution}








\end{document}